\documentclass{article}
\usepackage[utf8]{inputenc}

\usepackage{float}
\usepackage[T1]{fontenc}
\usepackage[english, russian]{babel}
\usepackage{amsmath}
\usepackage{amssymb}
\usepackage{amsthm}
\usepackage{setspace}
\usepackage{geometry}

\usepackage{mathtools}
\usepackage{tikz}
\usetikzlibrary{fit}
\tikzset{%
  highlight/.style={rectangle,rounded corners,fill=red!15,draw,fill opacity=0.5,thick,inner sep=0pt}
}
\newcommand{\tikzmark}[2]{
    \tikz[overlay,remember picture,baseline=(#1.base)] \node (#1) {#2};
}
%
\newcommand{\Highlight}[2]{%
    \tikz[overlay,remember picture]{
        \node[highlight,fit=(#1.north west) (#2.south east)] (#1) {};
    }
}


\usepackage{array}
\usetikzlibrary{positioning}
\definecolor{processblue}{cmyk}{0.96,0,0,0}

\geometry{
    a4paper,
    top=2.5cm,
    bottom=2.5cm,
    left=2.2cm,
    right=2.2cm
}

\onehalfspacing


\renewcommand{\thesection}{
    \underline{\textbf{№\arabic{section}}}
}

\renewcommand{\thesubsection}{
    \textbf{\arabic{section}.\asbuk{subsection}}
}

\newcommand{\mathInduction}[2]{
\underline{База:}\\
#1
\underline{Шаг индукции:}\\
#2
}

\newcommand{\mathMod}[2]{
    #1\%#2
}

\newcommand{\Z}[0]{
    \mathbb{Z}
}

\newcommand{\Q}[0]{
    \mathbb{Q}
}

\newcommand{\N}[0]{
    \mathbb{N}
}

\newcommand{\R}[0]{
    \mathbb{R}
}

\newcommand{\Inject}[1]{
    \stackrel{#1}{\lesssim}
}

\newcommand{\allsubsets}[1]{
    \mathcal{P}(#1)
}

\def\rddots#1{\cdot^{\cdot^{\cdot^{#1}}}}

\newcommand{\Knuth}[2]{
    \underbrace{#1^{#1^{\rddots#1}}}_{#2 \; копий \; #1}
}

\begin{document}
\section{Сформулируйте утверждение о связи размерностей ядра и образа линейного отображения.}
Пусть $\varphi : V_1 \longrightarrow V_2$ и $dim V_1 = n$.\\
Тогда $dim \; Ker \varphi + dim \; Im \varphi = n = dim V_1$.

\section{Дайте определения собственного вектора и собственного значения линейного оператора.}
Число $\lambda$ называется собственным числом линейного оператора $\varphi: V \longrightarrow V$,
если существует вектор $v \neq 0$ такой, что $\varphi(v) = \lambda v$.
При этом вектор $v$ называется собственным вектором оператора $\varphi$.

\section{Дайте определения характеристического уравнения и характеристического многочлена квадратной матрицы.}
Для произвольной квадратной матрицы $A$ многочлен вида $\chi_A(\lambda) = det(A - \lambda E)$ называется 
характеристическим многочленом, а уравнение $\chi_A(\lambda) = 0$ называется характеристическим
уравнением матрицы $A$.

\section{Сформулируйте утверждение о связи характеристического уравнения и спектра линейного оператора.}
\underline{Примечание} Здесь немного <<отсебятины>>, потому что переписывал с грубой формулы.
\\
Следующие условия эквивалентны:
\begin{enumerate}
    \item $\lambda$ - собственное значение линейного оператора $A$.
    \item $|A - \lambda E| = 0$(т.е. $\chi_A(\lambda) = 0$ или $\lambda$ является корнем характеристического многочлена $A$).
\end{enumerate}

\section{Дайте определение собственного подпространства.}
Собственным подпространством, отвечающим собственному значению $\lambda_i$ оператора $A$
называется множество:
$$ V_{\lambda_i} = \left\{x \in V \mid Ax = \lambda_i x \right\}$$

\section{Дайте определения алгебраической и геометрической кратности собственного значения. Какое неравенство их связывает?}
Алгебраической кратностью собственного значения называют его кратность как корня характеристического уравнения.
\\
\underline{Пример} $\chi_A(\lambda) = (\lambda - 5)^3 (\lambda - 6)^2 (\lambda + 3)$:
алгебраическая кратность собственного значения $\lambda = 5$ равна 3.
\\
Геометрической кратностью собственного значения называют размерность собственного подпространства $V_{\lambda_i}$.
\\
Геометрическая кратность собственного значения всегда положительна и не превосходит его алгебраической кратности.

\section{Каким свойством обладают собственные векторы линейного оператора, отвечающие различным собственным значениям?}
Пусть $\lambda_1, \ldots, \lambda_k$ - различные собственные значения линейного оператора $A$
($\forall i \neq j \;\; \lambda_i \neq \lambda_j$), 
а $v_1, \ldots, v_k$ - соответствующие им собственные вектора.
Тогда вектора $v_1, \ldots v_k$ линейно независимы.

\section{Сформулируйте критерий диагональности матрицы оператора.}
Матрица линейного оператора $A$ является диагональной в данном базисе $\iff$ все вектора этого базиса являются собственными векторами 
данного линейного оператора.

\section{Сформулируйте критерий диагонализируемости матрицы оператора с использованием понятия геометрической кратности.}
Матрица линейного оператора диагонализируема $\iff$ для любого его собственного значения геометрическая кратность
равна алгебраической кратности.

\section{Дайте определение жордановой клетки. Сформулируйте теорему о жордановой нормальной форме матрицы оператора.}
Жорданова клетка размера $m \times m$ соответствующего собственного значения $\lambda_i$ - матрица вида:
$$
J_m(\lambda_i) = \underbrace{
\begin{pmatrix}
\lambda_i & 1 & 0 & 0 & \cdots & 0\\
0 & \lambda_i & 1 & 0 & \cdots & 0\\
0 & 0 & \lambda_i & 1 & \cdots & 0\\
\vdots & \vdots & \vdots & \ddots & \vdots & \vdots\\
0 & 0 & 0 & 0 & \cdots & \lambda_i\\
\end{pmatrix}}_{\displaystyle m}
\left.\vphantom{\begin{pmatrix}
1\lambda_i & 1 & 0 & 0 & \cdots & 0\\
0 & \lambda_i & 1 & 0 & \cdots & 0\\
0 & 0 & \lambda_i & 1 & \cdots & 0\\
\vdots & \vdots & \vdots & \ddots & \vdots & \vdots\\
0 & 0 & 0 & 0 & \cdots & \lambda_i\\
\end{pmatrix}}\right\}m
$$
Жорданова нормальная форма матрицы линейного оператора - блочно-диагональная матрица
с Жордановыми клетками на диагонали.
$$
J = 
\begin{pmatrix}
    J_{m_1}(\lambda_1) & \cdots & \cdots & 0 \\
    0 & J_{m_2}(\lambda_2) & \cdots & 0 \\
    \vdots & \vdots & \ddots & \vdots \\
    0 & \cdots & \cdots & J_{m_s}(\lambda_s)\\
\end{pmatrix}
$$
\underline{Теорема} Любая матрица $A \in M_n(\mathbb{F})$ приводится заменой базиса к Жордановой нормальной форме 
над алгебраически замкнутым полем.

\section{ Выпишите формулу для количества жордановых клеток заданного размера.}
Пусть $q_h$ - число жордановых клеток размера $h \times h$ c $\lambda_i$ на диагонали.
Пример:
$$
\begin{aligned}
\begin{pmatrix}
    \tikzmark{A}{-3} & 0 & 0 & 0 & 0 & 0 \\
    0 & \tikzmark{B}{5} & 1 & 0 & 0 & 0 \\
    0 & 0 & \tikzmark{C}{5} & 0 & 0 & 0 \\
    0 & 0 & 0 & \tikzmark{D}{5} & 1 & 0 \\
    0 & 0 & 0 & 0 & 5 & 1 \\
    0 & 0 & 0 & 0 & 0 & \tikzmark{E}{5}
\end{pmatrix}
\Highlight{A}{A}
\Highlight{B}{C}
\Highlight{D}{E}
\\
\lambda_1 = -3, \lambda_2 = 5
\\
q_1(\lambda_1) = 1, q_2(\lambda_1) = 0,
\\
q_1(\lambda_2) = 0, q_2(\lambda_2) = 1, q_3(\lambda_2) = 1, q_4(\lambda_2) = 0
\\
\chi_A(\lambda) = (\lambda + 3)(\lambda - 5)^5
\end{aligned}
$$
\underline{Утверждение} Для любого собственного значения $\lambda_i \;\; q_h = r_{h + 1} - 2r_h + r_{h - 1}$, где 
$r_h = Rg(A - \lambda E)^h, r_0 = RgE$

\section{Сформулируйте теорему Гамильтона—Кэли.}
$\chi_A(A) = 0$, где $\chi_A$ - характеристический многочлен матрицы $A$.

\section{Дайте определение корневого подпространства.}
Корневым подпространством оператора $A$, соответствующим собственному значению $\lambda_i$
называют множество:
$K_{\lambda_i} = Ker(A - \lambda_i E)^{m_i}$, где $m_i$ - алгебраическая кратность
собственного значения.

\section{Дайте определение минимального многочлена линейного оператора.}
Минимальным многочленом линейного оператора $A$ называется многочлен $\mu_A(\lambda)$ такой, что:
\begin{enumerate}
    \item $\mu_A(A) = 0$
    \item $\mu_A \neq 0$
    \item Степень многочлена $\mu_A$ минимальна и старший коэффициент равен 1.
\end{enumerate}

\section{Дайте определение инвариантного подпространства.}
$L$ называется инвариантным подпространством линейного оператора
$A: V \longrightarrow V$, если $\forall x \in L \;\; Ax \in L$(т.е. $A(L) \subset L$)

\section{Дайте определение евклидова пространства.}
Евклидово пространство $\epsilon = (V, g)$ - линейное пространство над $\R$ с определенным скалярным произведение $g$,
где $g(x, y): V^2 \Longrightarrow \R$, удовлетворяющим следующим аксиомам:
\begin{enumerate}
    \item $\forall x, y, \in V \;\; g(x, y) = g(y, x)$
    \item $\forall x, y, z \in V \;\; g(x + y, z) = g(x, z) + g(y, z)$
    \item $g(\lambda x, y) = \lambda g(x, y)$
    \item $g(x, x) \geq 0$, причем $g(x, x) = 0 \iff x = 0$
\end{enumerate}

\section{Выпишите неравенства Коши–Буняковского и треугольника.}
\underline{Неравенство Коши-Коши–Буняковского} $\forall x, y \in \epsilon \;\; |(x, y)| \leq ||x|| \cdot ||y||$
\\
\underline{Неравенство треугольника} $\forall x, y \in \epsilon \;\; ||x + y|| \leq ||x|| + ||y||$

\section{Дайте определения ортогонального и ортонормированного базисов.}
Система векторов $v_1, \ldots, v_k$ называется:
\begin{itemize}
    \item Ортогональной, если $\forall i \neq j \;\; (v_i, v_j) = 0$
    \item Ортонормированной, если она ортогональна и $\forall i \;\; (v_i, v_i) = 1$
\end{itemize}
Если $k = dim V = n$, то $v_1, \ldots, v_k$ будет ортогональным базисом.
\\
Если рассмотреть $e_1 = \frac{v_1}{||v_1||}, \ldots, e_n = \frac{v_n}{||v_n||}$, то
получим ОНБ(ортонормированный базис).

\section{Дайте определение матрицы Грама.}
Пусть $a_1, \ldots, a_n$ - базис в $\epsilon$.
Тогда $g(x, y) = X^T \textup{Г} Y$, где $X, Y$ - столбцы координат векторов $x$
и $y$ в базисе $a_1, \ldots, a_n$. 
\\
$
\textup{Г} = 
\begin{pmatrix}
    (a_1, a_1) & \cdots & (a_1, a_n) \\
    \vdots & \ddots & \vdots \\
    (a_n, a_1) & \cdots & (a_n, a_n)
\end{pmatrix}
$ -
матрица Грама.

\section{Выпишите формулу для преобразования матрицы Грама при переходе к новому базису.}
Матрицы Грама двух базисов $e$ и $e'$ связаны следующим соотношением: $\textup{Г}'= U^T \textup{Г} U$,
где $U $– матрица перехода от $e$ к $e'$.

\section{Как меняется определитель матрицы Грама (грамиан) при применении процесса ортогонализации Грама–Шмидта?}
Определитель матрицы Грама не меняется при применении процесса ортогонализации Грама-Шмидта.

\section{Сформулируйте критерий линейной зависимости с помощью матрицы Грама.}
Пусть $Gr(a_1, \ldots, a_k) = det \textup{Г}$ - грамиан.
\\
Тогда
(вектора $a_1, \ldots, a_k$ линейно независимы)
$
\iff
Gr(a_1, \ldots, a_k) \neq 0
$

\section{Дайте определение ортогонального дополнения.}
Пусть $H \subseteq V$. Тогда множество
$H^{\perp} = \left\{x \in V \mid (x, y) = 0 \forall y \in H \right\}$
называется ортогональным дополнением.

\section{Дайте определения ортогональной проекции вектора на подпространство и ортогональной составляющей.}
$\forall x \in \epsilon \;\; x = y + z \;\;\;\; y \in H, z \in H^{\perp}$
\\
$y$ - ортогональная проекция $x$ на $H$
\\
$z$ - ортогональная составляющая $x$ относительно $H$

\section{
Выпишите формулу для ортогональной проекции вектора на подпространство,
заданное как линейная оболочка данного линейно независимого набора векторов.
}
Пусть $H = < a_1, \ldots , a_k >$ 
и вектора $a_1, \ldots , a_k$ линейно независимые.
\\
Тогда $\textup{пр}_H x = A (A^T A)^{-1} A^T x$,
где $A$ составлена из столбцов $a_1, \ldots a_k$.

\section{Выпишите формулу для вычисления расстояния с помощью определителей матриц Грама.}
Расстояние $p(x, P)$ между плоскостью(линейным многообразием)
$P = x_0 + L$, где $L = <a_1, \ldots , a_k>$ может быть найдено по формуле:
\\
$p^2(x, P) = \frac{det \textup{Г} (a_1, \ldots , a_k, x - x_0)}{det \textup{Г} (a_1, \ldots , a_k)} =
\frac{Gr(a_1, \ldots , a_k, x - x_0)}{Gr(a_1, \ldots , a_k)}$

\section{Дайте определение сопряженного оператора в евклидовом пространстве.}
Линейный оператор $A^*: \epsilon \longrightarrow \epsilon$ 
называется сопряженным к линейному оператору $A:\epsilon \longrightarrow \epsilon$,
если $\forall x, y \in \epsilon \;\; (Ax, y) = (x, A^*y)$

\section{Дайте определение самосопряженного (симметрического) оператора.}
Линейный оператор $A: \epsilon \longrightarrow \epsilon$ 
называется самосопржяенным(симметрическим), если
если $\forall x, y \in \epsilon \;\; (Ax, y) = (x, A^y)$,
т.е. $A^* = A$.

\section{Как найти матрицу сопряженного оператора в произвольном базисе?}
У любого линейного оператора $A: \epsilon \longrightarrow \epsilon$ 
существует и единственен сопряженный оператор 
$A^*: \epsilon \longrightarrow \epsilon$,
причем его матрицей будет матрица
$(A^*)_b = \textup{Г}^-1 (A)^T_b \textup{Г}$, где
$\textup{Г}$ - матрица грама базиса $b$.

\section{Каким свойством обладают собственные значения самосопряженного оператора?}
\begin{itemize}
    \item Все корни характеристического уравнения самосопряженного
    оператора являются действительными числами.
    \item Пусть $\lambda$ - собственное значение самосопряженного
    оператора $A$. Тогда алгебраическая кратность $\lambda$ равна
    геометрической кратности.
\end{itemize}

\section{Что можно сказать про собственные векторы самосопряженного оператора, отвечающие разным собственным значениям?}
Собственные вектора самосопряженного
линейного оператора, отвечающие 
разным собственным значениям, ортогональны.

\section{Сформулируйте определение ортогональной матрицы.}
Квадратную матрицу $M$ называют ортогональной,
если $M^T M = E$.

\section{Сформулируйте определение ортогонального оператора.}
Линейный оператор $A: \epsilon \longrightarrow \epsilon$ называется
ортогональным, если $\forall x, y \in \epsilon \;\; (Ax, Ay) = (x, y)$,
т.е. $A$ сохраняет скалярное произведение.

\section{Сформулируйте критерий ортогональности оператора, использующий его матрицу.}
Матрица линейненого оператора в ОНБ ортогональна $\iff A$ - ортогональный оператор. 

\section{Каков канонический вид ортогонального оператора? Сформулируйте теорему Эйлера.}
Для любого ортогонального оператора существует ОНБ, в ктором его матрица имеет
следующий блочно-диагональный вид:
\\
$$
\begin{pmatrix}
    A_{\varphi_1} & 0 & 0 & 0 & 0 & 0 & 0 & 0 & 0\\
    0 & \ddots & 0 & 0 & 0 & 0 & 0& 0 & 0\\
    0 & 0 & A_{\varphi_k} & 0 & 0 & 0 & 0 & 0 & 0\\
    0 & 0 & 0 & 1 & 0 & 0 & 0 & 0 & 0\\
    0 & 0 & 0 & 0 & \ddots & 0 & 0 & 0 & 0\\
    0 & 0 & 0 & 0 & 0 &  1 & 0 & 0 & 0\\
    0 & 0 & 0 & 0 & 0 & 0 & -1 & 0 & 0\\
    0 & 0 & 0 & 0 & 0 & 0 & 0 & \ddots & 0\\
    0 & 0 & 0 & 0 & 0 & 0 & 0 & 0 & -1
\end{pmatrix}
$$
Где $A_{\varphi_j} = 
\begin{pmatrix}
    \cos \varphi_j & -\sin \varphi_j\\
    \sin \varphi_j & \cos \varphi_j
\end{pmatrix}
$
\\
\underline{Следствие: теорема Эйлера}
\\
Для любого ортогонального преобразования в $R^3$
существует ОНБ, в котором его матрица имеет вид:
$$
\begin{pmatrix}
    \cos \varphi_j & -\sin \varphi_j & 0 \\
    \sin \varphi_j & \cos \varphi_j & 0\\
    0 & 0 & \pm 1 
\end{pmatrix}
$$
То есть любое ортогональное преобразование
в $R^3$ является или поворотом на некоторый угол
$\varphi$ вокруг оси, либо композици такого поворота с отражением.

\section{Сформулируйте теорему о существовании для самосопряженного оператора базиса из собственных векторов.}
Для любого самосопряженного оператора существует ОНБ, состоящий из собственных
векторов $A$. Матрица $A_e$ в этом базисе диагональна, а
на диагонали стоят собственные значения, повторяющиеся столько,
какова их алгебраическая кратность.

\section{Сформулируйте теорему о приведении квадратичной формы к диагональному виду при помощи ортогональной замены координат.}
Любую кввадратичую форму можно привести к диагональному виду при помощи ортогональной замены координат.

\section{Сформулируйте утверждение о QR-разложении.}
Пусть $A$ - квадратная матрица замера $n \times n$,
при этом столбцы $A_1, \ldots , A_n$ линейно независимы.
Тогда $A$ представима в виде $A = Q \cdot R$, где $Q$ - ортогональная
матрица, а $R$ - верхнетреугольная матрица.

\section{Сформулируйте теорему о сингулярном разложении.}
Для любой матрицы $A \in M_{mn}(\R)$ существует сингулярное разложение:
$$
A = V \mathbf{\Sigma} U^T
$$
Где $U \in M_n(\R)$ - ортогональная матрица,\\
$V \in M_m(\R)$ - ортогональная матрица,\\
$\mathbf{\Sigma} \in M_{mn}(\R)$ и $\mathbf{\Sigma}$ является
диагональной с числами $\varsigma_i \geq 0$ на диагонали(сингулярные числа).
При этом $\sigma_1 \geq \sigma_2 \geq \ldots \sigma_r > 0$

\section{Сформулируйте утверждение о полярном разложении.}
Любая матрица $A\in M_N(\R)$ представима в виде
$A = S \cdot U$, где $S$ - симметрическая матрица с положительными собственными
значениями, а $U$ - ортогональная матрица.

\section{Дайте определение сопряженного пространства.}
Пространством, сопряженным к линейному пространству $L$
называется множество линейных форм на нем с операциями
сложения и умножения на число.
\\
$(f_1 + f_2)(x) = f_1(x) + f_2(x) \;\; \forall  \in V$
\\
$(\lambda f)(x) =  \lambda f(x) \;\; \forall \lambda \in F$
\\
Обозначение:$L^* = Hom(L, F)$

\section{Выпишите формулу для преобразования координат ковектора при переходе к другому базису.}
TODO

\section{Дайте определение взаимных базисов.}
Базис $e = (e_1, \ldots , e_n)$ в линейном пространстве $L$ и
базис $f = (f^1, \ldots , f^n)$ в сопряженном пространстве $L^*$
называют взаимными, если
$
f^i(e_j) =
\begin{cases}
    1, i = j\\
    0, i \neq j
\end{cases} = \sigma^i_j
$

\section{Дайте определение биортогонального базиса.}
TODO

\section{Сформулируйте определение алгебры над полем. Приведите два примера.}
Пусть $A$ — векторное пространство над полем $K$,
снабженное операцией $A\times A\to A$, называемой умножением. 
Тогда $A$ является алгеброй над $K$, если для любых $x,y,z\in A, \; a,b\in K$
выполняются следующие свойства:
\begin{itemize}
    \item $(x+y)\cdot z=x\cdot z+y\cdot z$
    \item $x\cdot (y+z)=x\cdot y+x\cdot z$
    \item $(ax)\cdot (by)=(ab)(x\cdot y)$
\end{itemize}
Примеры: комплексные числа и кватернионы.    

\section{Сформулируйте определение тензора. Приведите два примера.}
Пусть $F$ - поле, $V$ - векторное пространство над $F$;
$V^*$ - сопряжженное к $V$; 
$p, q \in \N \cup \left\{ 0\right\}$
\\
Тогда любое полилинейное отображение
$f: \underbrace{V \times \ldots \times V}_p \times \underbrace{V \times \ldots \times V}_q \longrightarrow F$
называется тензором на $V$ типа $(p,q)$ и
валентности $p + q$.
\\
\underline{Примеры}
\begin{itemize}
    \item Тензор типа $(1, 0)$ - линейная функция на $V$, т.е. элементы $V^*$.
    \item Тензор типа $(2, 0)$ - билинейные формы на $V$.
    \item Тензор типа $(1, 1)$ можно интерпретировать как линейный оператор.
\end{itemize}

\section{Дайте определение эллипса как геометрического места точек. Выпишите его каноническое уравнение. Что такое эксцентриситет эллипса? В каких пределах он может меняться?}
Эллипсом называют геометрическое место точек,
сумма растояний от которых до двух данных точек, 
называемых \underline{фокусами}, постоянна.
\begin{center}
    Каноническое уравнение эллипса:
    $$
    \frac{x^2}{a^2} + \frac{y^2}{b^2} = 1
    $$
\end{center}
Число $\varepsilon = \frac{c}{a} = \frac{\sqrt{a^2 - b^2}}{a} = \sqrt{1 - \frac{b^2}{a^2}}$
называется эксцентриситетом эллипса.
\\
Эксцентриситет всегда лежит в полуинтервале $\left[0;1\right)$
и служит мерой <<сплюснутости>> эллипса.

\section{Дайте определение гиперболы как геометрического места точек. Выпишите её каноническое уравнение. Что такое эксцентриситет гиперболы? В каких пределах он может меняться?}
Гиперболой называют геометрическое место точек,
модуль разности расстояний от которых до двух данных точек,
называемых фокусами, постоянен.
\begin{center}
    Каноническое уравнение гиперболой:
    $$
    \frac{x^2}{a^2} - \frac{y^2}{b^2} = 1
    $$
\end{center}
Число $\varepsilon = \sqrt{1 + \frac{b^2}{a^2}}$
называется эксцентриситетом гиперболой.
\\
При этом $\varepsilon > 1$(при $\varepsilon \longrightarrow 1$
гипербола выражается в два луча) и характеризует
угол между асимптотами.

\section{Дайте определение параболы как геометрического места точек. Выпишите её каноническое уравнение.}
Параболой называется геометрическое место точек плоскости,
равноудаленных от данной точки $F$, называемой фокусом параболой,
и данной прямой, называемой ее директрисой.
\begin{center}
    Каноническое уравнение параболы:
    $$
    y^2 = 2px
    $$
\end{center}

\section{Дайте определение цилиндрической поверхности.}
Рассмотрим кривую $\gamma$, лежащую в некоторой плоскости $P$, и
прямую $L$, не лежащую в $P$. 
\\
Цилиндрической поверхностью называют множество
всех прямых, параллельных $L$ и пересекающих $\gamma$.

\section{Дайте определение линейчатой поверхности. Приведите три примера.}
Линейчатой называют поверхность, образованную
движением прямой линии.
\\
Примеры:
\begin{itemize}
    \item Цилиндр
    \item Гиперболический параболоид
    \item Линейчатый гиперболоид
\end{itemize}
\end{document}