\documentclass{article}
\usepackage[utf8]{inputenc}

\usepackage{float}
\usepackage[T1]{fontenc}
\usepackage[english, russian]{babel}
\usepackage{amsmath}
\usepackage{amssymb}
\usepackage{amsthm}
\usepackage{setspace}
\usepackage{geometry}
\usepackage{mathtools}
\usepackage{tikz}
\usepackage{array}
\usetikzlibrary{positioning}
\definecolor{processblue}{cmyk}{0.96,0,0,0}

\geometry{
    a4paper,
    top=2.5cm,
    bottom=2.5cm,
    left=2.2cm,
    right=2.2cm
}

\onehalfspacing


\renewcommand{\thesection}{
    \underline{\textbf{№\arabic{section}}}
}

\renewcommand{\thesubsection}{
    \textbf{\arabic{section}.\asbuk{subsection}}
}

\newcommand{\mathInduction}[2]{
\underline{База:}\\
#1
\underline{Шаг индукции:}\\
#2
}

\newcommand{\mathMod}[2]{
    #1\%#2
}

\newcommand{\Z}[0]{
    \mathbb{Z}
}

\newcommand{\Q}[0]{
    \mathbb{Q}
}

\newcommand{\N}[0]{
    \mathbb{N}
}

\newcommand{\R}[0]{
    \mathbb{R}
}

\newcommand{\Inject}[1]{
    \stackrel{#1}{\lesssim}
}

\newcommand{\allsubsets}[1]{
    \mathcal{P}(#1)
}

\def\rddots#1{\cdot^{\cdot^{\cdot^{#1}}}}

\newcommand{\Knuth}[2]{
    \underbrace{#1^{#1^{\rddots#1}}}_{#2 \; копий \; #1}
}

\begin{document}
\section{Сформулируйте утверждение о связи размерностей ядра и образа линейного отображения.}
Пусть $\varphi : V_1 \longrightarrow V_2$ и $dim V_1 = n$.\\
Тогда $dim \; Ker \varphi + dim \; Im \varphi = n = dim V_1$.

\section{Дайте определения собственного вектора и собственного значения линейного оператора.}
Число $\lambda$ называется собственным числом линейного оператора $\varphi: V \longrightarrow V$,
если существует вектор $v \neq 0$ такой, что $\varphi(v) = \lambda v$.
При этом вектор $v$ называется собственным вектором оператора $\varphi$.

\section{Дайте определения характеристического уравнения и характеристического многочлена квадратной матрицы.}
Для произвольной квадратной матрицы $A$ многочлен вида $\chi_A(\lambda) = det(A - \lambda E)$ называется 
характеристическим многочленом, а уравнение $\chi_A(\lambda) = 0$ называется характеристическим
уравнением матрицы $A$.

\section{Сформулируйте утверждение о связи характеристического уравнения и спектра линейного оператора.}
\underline{Примечание} Здесь немного <<отсебятины>>, потому что переписывал с грубой формулы.
\\
Следующие условия эквивалентны:
\begin{enumerate}
    \item $\lambda$ - собственное значение линейного оператора $A$.
    \item $|A - \lambda E| = 0$(т.е. $\chi_A(\lambda) = 0$ или $\lambda$ является корнем характеристического многочлена $A$).
\end{enumerate}

\section{Дайте определение собственного подпространства.}
Собственным подпространством, отвечающим собственному значению $\lambda_i$ оператора $A$
называется множество:
$$ V_{\lambda_i} = \left\{x \in V \mid Ax = \lambda_i x \right\}$$

\section{Дайте определения алгебраической и геометрической кратности собственного значения. Какое неравенство их связывает?}
Алгебраической кратностью собственного значения называют его кратность как корня характеристического уравнения.
\\
\underline{Пример} $\chi_A(\lambda) = (\lambda - 5)^3 (\lambda - 6)^2 (\lambda + 3)$:
алгебраическая кратность собственного значения $\lambda = 5$ равна 3.
\\
Геометрической кратностью собственного значения называют размерность собственного подпространства $V_{\lambda_i}$.
\\
Геометрическая кратность собственного значения всегда положительна и не превосходит его алгебраической кратности.

\section{Каким свойством обладают собственные векторы линейного оператора, отвечающие различным собственным значениям?}
Пусть $\lambda_1, \ldots, \lambda_k$ - различные собственные значения линейного оператора $A$
($\forall i \neq j \;\; \lambda_i \neq \lambda_j$), 
а $v_1, \ldots, v_k$ - соответствующие им собственные вектора.
Тогда вектора $v_1, \ldots v_k$ линейно независимы.

\section{Сформулируйте критерий диагональности матрицы оператора.}
Матрица линейного оператора $A$ является диагональной в данном базисе $\iff$ все вектора этого базиса являются собственными векторами 
данного линейного оператора.

\section{Сформулируйте критерий диагонализируемости матрицы оператора с использованием понятия геометрической кратности.}
Матрица линейного оператора диагонализируема $\iff$ для любого его собственного значения геометрическая кратность
равна алгебраической кратности.

\section{Дайте определение жордановой клетки. Сформулируйте теорему о жордановой нормальной форме матрицы оператора.}
Жорданова клетка размера $m \times m$ соответствующего собственного значения $\lambda_i$ - матрица вида:
$$
J_m(\lambda_i) = \underbrace{
\begin{pmatrix}
\lambda_i & 1 & 0 & 0 & \cdots & 0\\
0 & \lambda_i & 1 & 0 & \cdots & 0\\
0 & 0 & \lambda_i & 1 & \cdots & 0\\
\vdots & \vdots & \vdots & \ddots & \vdots & \vdots\\
0 & 0 & 0 & 0 & \cdots & \lambda_i\\
\end{pmatrix}}_{\displaystyle m}
\left.\vphantom{\begin{pmatrix}
1\lambda_i & 1 & 0 & 0 & \cdots & 0\\
0 & \lambda_i & 1 & 0 & \cdots & 0\\
0 & 0 & \lambda_i & 1 & \cdots & 0\\
\vdots & \vdots & \vdots & \ddots & \vdots & \vdots\\
0 & 0 & 0 & 0 & \cdots & \lambda_i\\
\end{pmatrix}}\right\}m
$$
Жорданова нормальная форма матрицы линейного оператора - блочно-диагональная матрица
с Жордановыми клетками на диагонали.
$$
J = 
\begin{pmatrix}
    J_{m_1}(\lambda_1) & \cdots & \cdots & 0 \\
    0 & J_{m_2}(\lambda_2) & \cdots & 0 \\
    \vdots & \vdots & \ddots & \vdots \\
    0 & \cdots & \cdots & J_{m_s}(\lambda_s)\\
\end{pmatrix}
$$
\underline{Теорема} Любая матрица $A \in M_n(\mathbb{F})$ приводится заменой базиса к Жордановой нормальной форме 
над алгебраически замкнутым полем.
\end{document}