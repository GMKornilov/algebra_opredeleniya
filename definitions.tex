\documentclass{article}
\usepackage[utf8]{inputenc}

\usepackage{float}
\usepackage[T1]{fontenc}
\usepackage[english, russian]{babel}
\usepackage{amsmath}
\usepackage{amssymb}
\usepackage{amsthm}
\usepackage{setspace}
\usepackage{geometry}

\usepackage{mathtools}
\usepackage{tikz}
\usetikzlibrary{fit}
\tikzset{%
  highlight/.style={rectangle,rounded corners,fill=red!15,draw,fill opacity=0.5,thick,inner sep=0pt}
}
\newcommand{\tikzmark}[2]{
    \tikz[overlay,remember picture,baseline=(#1.base)] \node (#1) {#2};
}
%
\newcommand{\Highlight}[2]{%
    \tikz[overlay,remember picture]{
        \node[highlight,fit=(#1.north west) (#2.south east)] (#1) {};
    }
}


\usepackage{array}
\usetikzlibrary{positioning}
\definecolor{processblue}{cmyk}{0.96,0,0,0}

\geometry{
    a4paper,
    top=2.5cm,
    bottom=2.5cm,
    left=2.2cm,
    right=2.2cm
}

\onehalfspacing


\renewcommand{\thesection}{
    \underline{\textbf{№\arabic{section}}}
}

\renewcommand{\thesubsection}{
    \textbf{\arabic{section}.\asbuk{subsection}}
}

\newcommand{\mathInduction}[2]{
\underline{База:}\\
#1
\underline{Шаг индукции:}\\
#2
}

\newcommand{\mathMod}[2]{
    #1\%#2
}

\newcommand{\Z}[0]{
    \mathbb{Z}
}

\newcommand{\Q}[0]{
    \mathbb{Q}
}

\newcommand{\N}[0]{
    \mathbb{N}
}

\newcommand{\R}[0]{
    \mathbb{R}
}

\newcommand{\Inject}[1]{
    \stackrel{#1}{\lesssim}
}

\newcommand{\allsubsets}[1]{
    \mathcal{P}(#1)
}

\def\rddots#1{\cdot^{\cdot^{\cdot^{#1}}}}

\newcommand{\Knuth}[2]{
    \underbrace{#1^{#1^{\rddots#1}}}_{#2 \; копий \; #1}
}

\begin{document}
\section{Сформулируйте утверждение о связи размерностей ядра и образа линейного отображения.}
Пусть $\varphi : V_1 \longrightarrow V_2$ и $dim V_1 = n$.\\
Тогда $dim \; Ker \varphi + dim \; Im \varphi = n = dim V_1$.

\section{Дайте определения собственного вектора и собственного значения линейного оператора.}
Число $\lambda$ называется собственным числом линейного оператора $\varphi: V \longrightarrow V$,
если существует вектор $v \neq 0$ такой, что $\varphi(v) = \lambda v$.
При этом вектор $v$ называется собственным вектором оператора $\varphi$.

\section{Дайте определения характеристического уравнения и характеристического многочлена квадратной матрицы.}
Для произвольной квадратной матрицы $A$ многочлен вида $\chi_A(\lambda) = det(A - \lambda E)$ называется 
характеристическим многочленом, а уравнение $\chi_A(\lambda) = 0$ называется характеристическим
уравнением матрицы $A$.

\section{Сформулируйте утверждение о связи характеристического уравнения и спектра линейного оператора.}
\underline{Примечание} Здесь немного <<отсебятины>>, потому что переписывал с грубой формулы.
\\
Следующие условия эквивалентны:
\begin{enumerate}
    \item $\lambda$ - собственное значение линейного оператора $A$.
    \item $|A - \lambda E| = 0$(т.е. $\chi_A(\lambda) = 0$ или $\lambda$ является корнем характеристического многочлена $A$).
\end{enumerate}

\section{Дайте определение собственного подпространства.}
Собственным подпространством, отвечающим собственному значению $\lambda_i$ оператора $A$
называется множество:
$$ V_{\lambda_i} = \left\{x \in V \mid Ax = \lambda_i x \right\}$$

\section{Дайте определения алгебраической и геометрической кратности собственного значения. Какое неравенство их связывает?}
Алгебраической кратностью собственного значения называют его кратность как корня характеристического уравнения.
\\
\underline{Пример} $\chi_A(\lambda) = (\lambda - 5)^3 (\lambda - 6)^2 (\lambda + 3)$:
алгебраическая кратность собственного значения $\lambda = 5$ равна 3.
\\
Геометрической кратностью собственного значения называют размерность собственного подпространства $V_{\lambda_i}$.
\\
Геометрическая кратность собственного значения всегда положительна и не превосходит его алгебраической кратности.

\section{Каким свойством обладают собственные векторы линейного оператора, отвечающие различным собственным значениям?}
Пусть $\lambda_1, \ldots, \lambda_k$ - различные собственные значения линейного оператора $A$
($\forall i \neq j \;\; \lambda_i \neq \lambda_j$), 
а $v_1, \ldots, v_k$ - соответствующие им собственные вектора.
Тогда вектора $v_1, \ldots v_k$ линейно независимы.

\section{Сформулируйте критерий диагональности матрицы оператора.}
Матрица линейного оператора $A$ является диагональной в данном базисе $\iff$ все вектора этого базиса являются собственными векторами 
данного линейного оператора.

\section{Сформулируйте критерий диагонализируемости матрицы оператора с использованием понятия геометрической кратности.}
Матрица линейного оператора диагонализируема $\iff$ для любого его собственного значения геометрическая кратность
равна алгебраической кратности.

\section{Дайте определение жордановой клетки. Сформулируйте теорему о жордановой нормальной форме матрицы оператора.}
Жорданова клетка размера $m \times m$ соответствующего собственного значения $\lambda_i$ - матрица вида:
$$
J_m(\lambda_i) = \underbrace{
\begin{pmatrix}
\lambda_i & 1 & 0 & 0 & \cdots & 0\\
0 & \lambda_i & 1 & 0 & \cdots & 0\\
0 & 0 & \lambda_i & 1 & \cdots & 0\\
\vdots & \vdots & \vdots & \ddots & \vdots & \vdots\\
0 & 0 & 0 & 0 & \cdots & \lambda_i\\
\end{pmatrix}}_{\displaystyle m}
\left.\vphantom{\begin{pmatrix}
1\lambda_i & 1 & 0 & 0 & \cdots & 0\\
0 & \lambda_i & 1 & 0 & \cdots & 0\\
0 & 0 & \lambda_i & 1 & \cdots & 0\\
\vdots & \vdots & \vdots & \ddots & \vdots & \vdots\\
0 & 0 & 0 & 0 & \cdots & \lambda_i\\
\end{pmatrix}}\right\}m
$$
Жорданова нормальная форма матрицы линейного оператора - блочно-диагональная матрица
с Жордановыми клетками на диагонали.
$$
J = 
\begin{pmatrix}
    J_{m_1}(\lambda_1) & \cdots & \cdots & 0 \\
    0 & J_{m_2}(\lambda_2) & \cdots & 0 \\
    \vdots & \vdots & \ddots & \vdots \\
    0 & \cdots & \cdots & J_{m_s}(\lambda_s)\\
\end{pmatrix}
$$
\underline{Теорема} Любая матрица $A \in M_n(\mathbb{F})$ приводится заменой базиса к Жордановой нормальной форме 
над алгебраически замкнутым полем.

\section{ Выпишите формулу для количества жордановых клеток заданного размера.}
Пусть $q_h$ - число жордановых клеток размера $h \times h$ c $\lambda_i$ на диагонали.
Пример:
$$
\begin{aligned}
\begin{pmatrix}
    \tikzmark{A}{3} & 0 & 0 & 0 & 0 & 0 \\
    0 & \tikzmark{B}{5} & 1 & 0 & 0 & 0 \\
    0 & 0 & \tikzmark{C}{5} & 0 & 0 & 0 \\
    0 & 0 & 0 & \tikzmark{D}{5} & 1 & 0 \\
    0 & 0 & 0 & 0 & 5 & 1 \\
    0 & 0 & 0 & 0 & 0 & \tikzmark{E}{5}
\end{pmatrix}
\Highlight{A}{A}
\Highlight{B}{C}
\Highlight{D}{E}
\\
\lambda_1 = -3, \lambda_2 = 5
\\
q_1(\lambda_1) = 1, q_2(\lambda_1) = 0,
\\
q_1(\lambda_1) = 0, q_2(\lambda_2) = 1, q_3(\lambda_2) = 1, q_4(\lambda_2) = 0
\\
\chi_A(\lambda) = (\lambda + 3)(\lambda - 5)^5
\end{aligned}
$$
\underline{Утверждение} Для любого собственного значения $\lambda_i \;\; q_h = r_{h + 1} - 2r_h + r_{h - 1}$, где 
$r_h = Rg(A - \lambda E)^h, r_0 = RgE$

\section{Сформулируйте теорему Гамильтона—Кэли.}
$\chi_A(A) = 0$, где $\chi_A$ - характеристический многочлен матрицы $A$.

\section{Дайте определение корневого подпространства.}
Корневым подпространством оператора $A$, соответствующим собственному значению $\lambda_i$
называют множество:
$K_{\lambda_i} = Ker(A - \lambda_i E)^{m_i}$, где $m_i$ - алгебраическая кратность
собственного значения.

\section{Дайте определение минимального многочлена линейного оператора.}
Минимальным многочленом линейного оператора $A$ называется многочлен $\mu_A(\lambda)$ такой, что:
\begin{enumerate}
    \item $\mu_A(A) = 0$
    \item $\mu_A \neq 0$
    \item Степень многочлена $\mu_A$ минимальна и старший коэффициент равен 1.
\end{enumerate}

\section{Дайте определение инвариантного подпространства.}
$L$ называется инвариантным подпространством линейного оператора
$A: V \longrightarrow V$, если $\forall x \in L \;\; Ax \in L$(т.е. $A(L) \subset L$)

\section{Дайте определение евклидова пространства.}
Евклидово пространство $\epsilon = (V, g)$ - линейное пространство над $\R$ с определенным скалярным произведение $g$,
где $g(x, y): V^2 \Longrightarrow \R$, удовлетворяющим следующим аксиомам:
\begin{enumerate}
    \item $\forall x, y, \in V \;\; g(x, y) = g(y, x)$
    \item $\forall x, y, z \in V \;\; g(x + y, z) = g(x, z) + g(y, z)$
    \item $g(\lambda x, y) = \lambda g(x, y)$
    \item $g(x, x) \geq 0$, причем $g(x, x) = 0 \iff x = 0$
\end{enumerate}

\section{Выпишите неравенства Коши–Буняковского и треугольника.}
\underline{Неравенство Коши-Коши–Буняковского} $\forall x, y \in \epsilon \;\; |(x, y))| \leq ||x|| \cdot ||y||$
\\
\underline{Неравенство треугольника} $\forall x, y \in \epsilon \;\; ||x + y|| \leq ||x|| + ||y||$

\section{Дайте определения ортогонального и ортонормированного базисов.}
Система векторов $v_1, \ldots, v_k$ называется:
\begin{itemize}
    \item Ортогональной, если $\forall i \neq j \;\; (v_i, v_j) = 0$
    \item Ортонормированной, если она ортогональна и $\forall i \;\; (v_i, v_i) = 1$
\end{itemize}
Если $k = dim V = n$, то $v_1, \ldots, v_k$ будет ортогональным базисом.
\\
Если рассмотреть $e_1 = \frac{v_1}{||v_1||}, \ldots, e_n = \frac{v_n}{||v_n||}$, то
получим ОНБ(ортонормированный базис).

\section{Дайте определение матрицы Грама.}
Пусть $a_1, \ldots, a_n$ - базис в $\epsilon$.
Тогда $g(x, y) = X^T \textup{Г} Y$, где $X, Y$ - столбцы координат векторов $x$
и $y$ в базисе $a_1, \ldots, a_n$. 
\\
$
\textup{Г} = 
\begin{pmatrix}
    (a_1, a_1) & \cdots & (a_1, a_n) \\
    \vdots & \ddots & \vdots \\
    (a_n, a_1) & \cdots & (a_n, a_n)
\end{pmatrix}
$ -
матрица Грама.

\section{Выпишите формулу для преобразования матрицы Грама при переходе к новому базису.}
Матрицы Грама двух базисов $e$ и $e'$ связаны следующим соотношением: $\textup{Г}'= U^T \textup{Г} U$,
где $U $– матрица перехода от $e$ к $e'$.

\section{Как меняется определитель матрицы Грама (грамиан) при применении процесса ортогонализации Грама–Шмидта?}
Определитель матрицы Грама не меняется при применении процесса ортогонализации Грама-Шмидта.

\section{Сформулируйте критерий линейной зависимости с помощью матрицы Грама.}
Пусть $Gr(a_1, \ldots, a_k) = det \textup{Г}$ - грамиан.
\\
Тогда
(вектора $a_1, \ldots, a_k$ линейно независимы)
$
\iff
Gr(a_1, \ldots, a_k) \neq 0
$

\section{Дайте определение ортогонального дополнения.}
Пусть $H \subseteq V$. Тогда множество
$H^{\perp} = \left\{x \in V \mid (x, y) = 0 \forall y \in H \right\}$
называется ортогональным дополнением.

\section{Дайте определения ортогональной проекции вектора на подпространство и ортогональной составляющей.}
$\forall x \in \epsilon \;\; x = y + z \; y \in H, z \in H^{\perp}$
\\
$y$ - ортогональная проекция $x$ на $H$
\\
$z$ - ортогональная составляющая $x$ относительно $H$

\section{
Выпишите формулу для ортогональной проекции вектора на подпространство,
заданное как линейная оболочка данного линейно независимого набора векторов.
}
Пусть $H = < a_1, \ldots , a_k >$ 
и вектора $a_1, \ldots , a_k$ линейно независимые.
\\
Тогда $\textup{пр}_H x = A (A^T A)^{-1} A^T x$,
где $A$ составлена из столбцов $a_1, \ldots a_k$.
\end{document}